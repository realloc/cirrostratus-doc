\documentclass[a4paper,12pt]{article}
\usepackage[utf8]{inputenc}
\usepackage[russian]{babel}
\pagestyle{headings}
\usepackage[pdftex,unicode]{hyperref}
\usepackage[T2A]{fontenc}
%\usepackage{indentfirst}
\usepackage{graphicx}
\usepackage{cmap}

\title{Cirrostratus Functional Specification}
\author{Stanislav Bogatyrev (realloc@realloc.spb.ru)}
\date{\today}

\begin{document}
\maketitle
\tableofcontents

\section{About}
\subsection{Introduction}
Описание Cirrostratus и его предполагаемого устройства. Некое подобие
фанспеки, из которой можно будет сделать настоящий формализованный
документ, если в том будет потребность. Хорошо если спека в какой-то
момент будет переведена на английский как на основной язык.

\subsection{Open Issues}
Best here is to maintain this as a list of issues which have been
raised against the spec.  As they are closed, indicate the resolution,
i.e. “changed section x.x.x to address this issue”, “issue noted but
deferred to a later release”, etc.

\subsection{Terminology}

\subsection{References}
References to external information sources.

\section{High-level Requirements Overview}
\subsection{High- level Requirements}

\begin{tabular}{|c|l|c|}
\hline
ID & Description & Supported \\ \hline
REQ\_0 & AoE compatible export for clients & No\\
\hline
\end{tabular}


\subsection{Assumptions}
\begin{tabular}{|c|l|c|}
\hline
ID & Description & Supported \\ \hline
REQ\_0A & We still can use AoE extensions  & No\\
\hline
\end{tabular}


\subsection{Limitations}
\begin{tabular}{|c|l|c|}
\hline
ID & Description & Supported \\ \hline
REQ\_0L & Shelf and slot numbers must confirm AoE spec & No\\
\hline
\end{tabular}


\section{Detailed Functionality}
\subsection{Functional description}
\section{Functional Requirements}
% use-cases here
\section{Non-functional Requirements}


\end{document}
